\documentclass[12pt]{article}
\usepackage[letterpaper, margin=1in]{geometry}
\usepackage{amsmath}
\usepackage{amsthm}
\usepackage{amssymb}

\newenvironment*{solution}{\begin{proof}[Solution]}{\end{proof}}
\renewcommand*{\qedsymbol}{\(\blacksquare\)}

\title{MTH 4320 Homework 4}
\author{Yaohui Wu}
\date{March 6, 2024}

\begin{document}
\maketitle
\section*{Problem 1}
\begin{solution}
    The edges of the breadth-first search (BFS) tree of the graph \(G\)
    starting from root \(g\) are \(e_1=\{g,b\},e_2=\{g,i\},e_3=\{g,j\},e_4=\{
        b,a\},e_5=\{b,c\},e_6=\{i,d\},e_7=\{i,f\},e_8=\{j,e\},e_9=\{j,h\}\) in
        the order from first to last respectively, that which was to be
        demonstrated.
\end{solution}
\section*{Problem 2}
\begin{solution}
    The algorithm is
    \begin{enumerate}
        \item Let the vertex of every squirrel be the root of a BFS tree.
        \item Run a modified BFS alogrithm for every squirrel to find all vertices
        that each root can reach using at most 5 edges.
        \item If there is a common vertex in the BFS trees of all five
        squirrels then return true else return false.
    \end{enumerate}
    This is a modified BFS alogrithm so we need to visit every vertex and edge
    that are necessary to check. Visiting all vertices takes \(O(|V|)\) time,
    visiting all edges takes \(O(|E|)\) time, and all other operations takes
    \(O(1)\) time. Therefore, the running time is \(O(|V|+|E|)\).
\end{solution}
\section*{Problem 3}
\begin{solution}
    We can change the graph \(G\) to \(G'\) by adding an vertex in the middle
    of every blue edge s.t. that \(G'\) has all uncolored edges of weight 1.
    Then we can apply the BFS algorithm on \(G'\) starting from root \(s\) to
    find the shortest paths tree in \(G'\). The shortest paths tree from the
    vertex \(s\) in \(G'\) will be the same as the shortest paths tree in \(G
    \) after we remove the extra vertices. The number of vertices we can add
    in \(G'\) is at most \(|E(G)|\) so \(G'\) has \(|V(G)|+|E(G)|\) vertices.
    Hence the number of edges in \(G'\) is at most \(2\cdot|E(G)|\). Since we
    are using the BFS algorithm on \(G'\),the running time is
    \begin{align*}
        O(|V(G')|+|E(G')|) &=O(|V(G)|+|E(G)|+2\cdot|E(G)|)\\ &=O(|V(G)|+|E(G)|
        )\\ &=O(|V(G)+E(G)|)
    \end{align*}
    Therefore, the time complexity of the algorithm is \(O(|V+E|)\).
\end{solution}
\section*{Problem 4}
\begin{solution}
    The algorithm is
    \begin{enumerate}
        \item Start from any vertex \(v\) and run the BFS algorithm on graph
        \(G\).
        \item If \(G\) is connected then the set of all of the vertices in
        \(G\) is the largest commune.
        \item If \(G\) is not connected then we run BFS for each component of
        \(G\). We update the largest commune with the set of all vertices in
        the component that has the largest number of vertices.
    \end{enumerate}
    The BFS algorithm visits all of the vertices and edges in the graph so the
    time complexity is \(O(|V|+|E|)\).
\end{solution}
\end{document}