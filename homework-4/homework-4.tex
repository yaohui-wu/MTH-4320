\documentclass[12pt]{article}
\usepackage[letterpaper, margin=1in]{geometry}
\usepackage{amsthm}
\usepackage{amssymb}

\newenvironment*{solution}{\begin{proof}[solution]}{\end{proof}}
\renewcommand*{\qedsymbol}{\(\blacksquare\)}

\title{MTH 4320 Homework 4}
\author{Yaohui Wu}
\date{February 29, 2024}

\begin{document}
\maketitle
\section*{Problem 1}
\begin{solution}
    The edges of the breadth-first search (BFS) tree of the graph \(G\)
    starting from root \(g\) are \(e_1=\{g,b\},e_2=\{g,i\},e_3=\{g,j\},e_4=\{
        b,a\},e_5=\{b,c\},e_6=\{i,d\},e_7=\{i,f\},e_8=\{j,e\},e_9=\{j,h\}\) in
        the order from first to last respectively, that which was to be
        demonstrated.
\end{solution}
\section*{Problem 2}
\begin{solution}
    The algorithm is a modified BFS.
    \begin{enumerate}
        \item Let the vertex of every squirrel be the root of a BFS tree.
        \item Run the modified BFS  for every squirrel to find all vertices
        that each root can reach using at most 5 edges.
        \item If there is a common vertex in the BFS trees of all five
        squirrels then return true else return false.
    \end{enumerate}
    This is a modified BFS alogrithm so we need to visit every vertex and edge
    that are necessary to check. Visiting all vertices takes \(O(|V|)\) time,
    visiting all edges takes \(O(|E|)\) time, and all other operations takes
    \(O(1)\) time. Therefore, the running time is \(O(|V|+|E|)\).
\end{solution}
\end{document}