\documentclass[12pt]{article}
\usepackage[letterpaper, margin=1in]{geometry}
\usepackage{mlmodern}
\usepackage{amsthm}
\usepackage{amssymb}

\newenvironment*{solution}{\begin{proof}[Solution]}{\end{proof}}
\renewcommand*{\qedsymbol}{\(\blacksquare\)}

\title{MTH 4320 Homework 7}
\author{Yaohui Wu}
\date{April 3, 2024}

\begin{document}
\maketitle
\section*{Problem 1}
\begin{solution}
    We run Prim's algorithm to find a minimum spanning tree (MST) of the graph
    with root \(d\) then the edges added to the MST are \(\{d,e\},\{e,g\},\{e,
    h\},\{b,e\},\{a,b\},\{a,c\},\{c,f\}\) from first to last respectively.
\end{solution}
\section*{Problem 2}
\begin{solution}
    We can use a modified Prim's algorithm to find the maximum spanning tree
    of \(G\). We can find the greatest edge \(e\) instead of the lightest edge
    while keeping everything else in the algorithm the same. We can multiply
    the weight of every edge by \(-1\) and we can use \(-\infty\) instead of
    \(\infty\) in the priority queue. Therefore, when we remove a vertex with
    the highest priority in the queue we will choose the edge with the highest
    weight in the original graph. The time complexity of the modified Prim's
    algorithm is \(O(|V|\log|V|+|E|)\).
\end{solution}
\section*{Problem 3}
\begin{solution}
    Since the weight of each edge is either 0 or 5 thus we can directly choose
    the lightest edge incidental to each vertex. Let \(u\) be an arbitrary
    vertex in \(G\), we check all of its incidental edges. If an edge has
    weight 0 then we add that edge to connect the adjacent vertex \(v\). If
    all the edges have weight 5 then we can add any edge. Hence, we can make
    sure that we select the lightest edge to connect each vertex in \(G\). We
    run operations with \(O(1)\) time for \(|V|\) vertices and \(|E|\) edges
    so the time complexity of the algorithm is \(O(|V|+|E|)\).
\end{solution}
\section*{Problem 4}
\begin{solution}
    We can use BFS to visit the vertices and edges in \(G\) to find \(e\) and
    this takes \(O(|V|+|E|)\) time. We can compare the weight of \(e\) to the
    edge with the same endpoints in the MST. If the weight is lighter, then we
    replace the edge with \(e\) in the new MST else the new MST is the same as
    \(T\). The time complexity of the algorithm is \(O(|V|+|E|)\).
\end{solution}
\end{document}