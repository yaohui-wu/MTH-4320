\documentclass[12pt]{article}
\usepackage[letterpaper, margin=1in]{geometry}
\usepackage{amsthm}
\usepackage{amssymb}

\newenvironment*{solution}{\begin{proof}[Solution]}{\end{proof}}
\renewcommand*{\qedsymbol}{\(\blacksquare\)}

\title{MTH 4320 Homework 5}
\author{Yaohui Wu}
\date{March 13, 2024}

\begin{document}
\maketitle
\section*{Problem 1}
\begin{solution}
    Let \textit{sum} be an attribute of the stack and \textit{sum} is 0 when
    we create the stack. If the stack is empty and we push an element then we
    add that number to \textit{sum} to update the sum of the stack. Similarly,
    if the stack is not empty and we pop an element then we subtract it from
    \textit{sum} to update the sum. If \(S_1\) is empty and we push an element
    then \textit{max} is that element which we also push it to \(S_2\). Let \(
    S_1\) be the stack storing all of the elements and let \(S_2\) be a stack
    that stores the values of \textit{max}. Every time we push an element we
    compare it with \textit{max} and update \textit{max} if the new element is
    greater then push the new element to both \(S_1\) and \(S_2\). Similarly,
    when we pop an element from \(S_1\) we check if it is the top element in
    \(S_2\) which is \textit{max}. If it is \textit{max} then remove the
    element from \(S_1\) and \(S_2\). The new \textit{max} of \(S_1\) is the
    new top element in \(S_2\). Since all operations take \(O(1)\) time to
    update \textit{sum} and \textit{max}, hence the time complexity of the
    operations \textit{sum} and \textit{max} are \(O(1)\).
\end{solution}
\section*{Problem 2}
\begin{solution}
    We can implement the disjoint-set data structure using trees. We implement
    Create(\(x\)) by creating a new tree with a single vertex \(x\) and this
    takes \(O(1)\) time. The root of the tree has the name of the set and
    number \(x\). The leaves are all of the elements in the same set. We
    implement Find(\(x\)) by iterating over all vertices of all the trees so
    this takes \(O(n)\) time. We implement Union(\(Y, Z\)) by attaching the
    root of \(Z\) to \(Y\) and update its name of the set to \(Y\) so the time
    complexity is \(O(1)\).
\end{solution}
\section*{Problem 3}
\begin{solution}
    
\end{solution}
\section*{Problem 4}
\begin{solution}
    Let \(G\) be a graph where the vertices represent the entries of the
    Sudoku grid. There is an edge between two vertices if they are in the same
    row or the same column. In addition, there is an edge between two vertices
    in the same subgrid. The graph coloring problem is to color the vertices
    with nine different colors s.t. no adjacent vertices have the same color.
\end{solution}
\end{document}