\documentclass[12pt, letterpaper]{article}
\usepackage[margin=1in]{geometry}
\usepackage{amsthm}
\usepackage{amssymb}

\newenvironment{solution}{\begin{proof}[Solution]\hfill\\}{\end{proof}}
\renewcommand*{\qedsymbol}{\(\blacksquare\)}

\title{MTH 4320 Homework 1}
\author{Yaohui Wu}
\date{\today \\ Due by February 7, 2024}

\begin{document}
\maketitle
\tableofcontents
\section{Problem 1}
\begin{solution}
    We have a triple nested loop and for the outermost loop we run \(n-1\) operations
    then we run \(n-1\) operations in the inner loop for every operation in the outer loop
    and so on. The running time is \((n-1)(n-1)(n-1)=O(n^3)\).
\end{solution}

\section{Problem 2}
\begin{solution}
    We have \(n-2\) operations from the outer loop where \(n\) is the input number \textit{num}. Then we call the \textit{is\_prime} function twice for every operation
    and the function runs at most \(n-2\) operations every time. The running time is \(2(n-2)(n-2)=O(n^2)\).
\end{solution}

\section{Problem 3}
\begin{solution}
    \begin{enumerate}
        \item Store the first element of the input list \(L\) in a variable \(M\).
        \item For every element in \(L\):
        \begin{itemize}
            \item If it is greater than \(M\) then update \(M\) with that element.
        \end{itemize}
        \item We found the first largest element \(M\) in \(L\).
        \item Let \(N:=\) the first element of \(L\).
        \item For every element in \(L\):
        \begin{itemize}
            \item If it is greater than \(N\) but less than \(M\) then update \(N\) with that element.
        \end{itemize}
        \item We found the second largest element \(N\) in \(L\).
        \item Let \(M:=N\) and \(N:=\) the first element of \(L\).
        \item Repeat steps 2-6 until we found the 10th largest element \(M\).
    \end{enumerate}
    The running time is \(10n=O(n)\).
\end{solution}

\section{Problem 4}
\begin{solution}
    
\end{solution}
\end{document}