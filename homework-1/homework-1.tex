\documentclass[12pt, letterpaper]{article}
\usepackage[margin=1in]{geometry}
\usepackage{amsthm}
\usepackage{amssymb}

\newenvironment{solution}{\begin{proof}[Solution]}{\end{proof}}
\renewcommand*{\qedsymbol}{\(\blacksquare\)}

\title{MTH 4320 Homework 1}
\author{Yaohui Wu}
\date{\today \\ Due by February 7, 2024}

\begin{document}
\maketitle
\tableofcontents
\section{Problem 1}
\begin{solution}
    We have a triple nested loop and for the outermost loop we run \(n-1\) operations
    then we run \(n-1\) operations in the inner loop for every operation in the outer loop
    and so on. The running time is \((n-1)(n-1)(n-1)=O(n^3)\).
\end{solution}

\section{Problem 2}
\begin{solution}
    We have \(n-2\) operations from the outer loop where \(n\) is the input number \textit{num}. Then we call the \textit{is\_prime} function twice for every operation
    and the function runs at most \(n-2\) operations every time. The running time is \(2(n-2)(n-2)=O(n^2)\).
\end{solution}

\section{Problem 3}
\begin{solution}
    
\end{solution}

\section{Problem 4}
\begin{solution}
    
\end{solution}
\end{document}