\documentclass[12pt]{article}
\usepackage[letterpaper, margin=1in]{geometry}
\usepackage{amsthm}
\usepackage{amssymb}

\newenvironment{solution}{\begin{proof}[Solution]}{\end{proof}}
\renewcommand*{\qedsymbol}{\(\blacksquare\)}

\title{MTH 4320 Homework 3}
\author{Yaohui Wu}
\date{February 21, 2024}

\begin{document}
\maketitle
\section*{Problem 1}
\begin{solution}
    Let the graph on the left be \(G\) and the graph of the drawing be \(H\).
    Notice that the set of vertices \(\{a,c,f,g,h\}\) in \(G\) is an independent
    set, where no two vertices in the set are adjacent. We can label the vertices
    in \(H\) as \(\{a,c,f,h\}\) on the left side and \(\{g,e,d,b\}\) on the right
    side from top to bottom respectively. Then we observe that vertex \(a\) is
    adjacent to the vertices \(\{b,d,e\}\) denoted by \(a\sim\{b,d,e\}\).
    Similarly, we observe that \(c\sim\{b,d,g\},f\sim\{b,e,g\}\), and \(h\sim\{d,e,g\}\).
    Hence we have that any two vertices with the same labels are adjacent in \(G\)
    if and only if they are adjacent in \(H\). Therefore, it is shown that \(G\)
    and \(H\) are identical or \(G\cong H\), \(G\) is isomorphic to \(H\).
\end{solution}

\section*{Problem 2}
\begin{solution}
    The algorithm is
    \begin{enumerate}
        \item Label the vertices in graph \(G\) as \(V=\{v_1,\dots,v_n\}\).
        This takes \(O(|V|)\) time.
        \item For every vertex \(v_i\) in \(V\) starting from \(v_1\)
        \begin{itemize}
            \item Assign a direction to every edge incidental to \(v_i\) that
            has not been assigned a direction.
            \item If the other endpoint has a greater value of \(i\) then we
            assign the edge to that direction else we assign it to the opposite
            direction.
        \end{itemize}
        \item Labeling all of the edges takes \(O(|E|)\) time.
    \end{enumerate}
    The algorithm ensures that the directions of all edges go to the vertices
    with a label of higher value. Therefore, there must not be any cycle in
    the directed graph. The running time of the algorithm is
    \(O(|V|)+O(|E|)=O(|V|+|E|)\).
\end{solution}

\section*{Problem 3}
\begin{solution}
    For a binary sequence of length \(k=3\), \(G\) can be a cube with 8 vertices
    being the corners having edges connecting the other 3 adjacent corners. In
    general, for any positive integer \(n\) we have that \(G(n)\) is a connected
    graph with only one component. We can prove this proposition using proof by
    induction. Let the statement be \(P(k)\) for all positive integer \(k\)
    where k is the length of the binary sequence. \\
    Base case: For \(k=1\) we have two vertices 0 and 1 connected by an edge so
    \(P(1)\) is true. \\
    Induction steps: Assume that \(P(k)\) is true so \(G(k)\) is connected.
    When we add one more bit to the binary sequence of \(G(k)\), we can choose
    any position and insert the same bit to maintain the graph \(G(k)\). Then
    we can change the new bit of each new binary sequence to create some new
    vertices connected to the existing vertices of \(G(k)\) to form the graph
    \(G(k+1)\). Hence \(G(k+1)\) is constructed from \(G(k)\) and \(G(k)\) is
    connected so \(G(k+1)\) is connected and \(P(k+1)\) is true. \\
    Therefore, it is proved by mathematical induction that for any positive
    integer \(n\) the graph \(G(n)\) is connected so it has only one component.
\end{solution}
\end{document}