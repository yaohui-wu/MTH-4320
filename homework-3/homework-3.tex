\documentclass[12pt]{article}
\usepackage[letterpaper, margin=1in]{geometry}
\usepackage{amsthm}
\usepackage{amssymb}

\newenvironment{solution}{\begin{proof}[Solution]}{\end{proof}}
\renewcommand*{\qedsymbol}{\(\blacksquare\)}

\title{MTH 4320 Homework 3}
\author{Yaohui Wu}
\date{February 18, 2024}

\begin{document}
\maketitle
\section*{Problem 1}
\begin{solution}
    Let the graph on the left be \(G\) and the graph of the drawing be \(H\).
    Notice that the set of vertices \(\{a,c,f,g,h\}\) in \(G\) is an independent
    set where no two vertices in the set are adjacent. We can label the vertices
    in \(H\) as \(\{a,c,f,h\}\) on the left side and \(\{g,e,d,b\}\) on the right
    side from top to bottom respectively. Then we have the vertex \(a\) is adjacent
    to the vertices \(\{b,d,e\}\) denoted by \(a\sim\{b,d,e\}\). Similarly, we have
    \(c\sim\{b,d,g\},f\sim\{b,e,g\}\), and \(h\sim\{d,e,g\}\). Hence we have that
    any two vertices with the same labels are adjacent in \(G\) if and only if
    they are adjacent in \(H\). Therefore, it is shown that \(G\) and \(H\) are
    identical or \(G\cong H\), \(G\) is isomorphic to \(H\).
\end{solution}
\section*{Problem 2}
\section*{Problem 3}
\end{document}