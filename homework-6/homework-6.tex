\documentclass[12pt]{article}
\usepackage[letterpaper, margin=1in]{geometry}
\usepackage{amsthm}
\usepackage{amssymb}

\newenvironment*{solution}{\begin{proof}[Solution]}{\end{proof}}
\renewcommand*{\qedsymbol}{\(\blacksquare\)}

\title{MTH 4320 Homework 6}
\author{Yaohui Wu}
\date{March 20, 2024}

\begin{document}
\maketitle
\section*{Problem 1}
\begin{solution}
    If the cell is empty then we can insert with constant time. If the cell is
    not empty then there is a collision so we need to iterate over the other
    cells to find an empty one. Therefore, the time complexity of insertion
    without collision is \(O(1)\) and with collision is \(O(n)\). If the
    element is in cell \(i\) then we can search in constant time. If it is not
    in cell \(i\) then we need to iterate over the other cells to find it.
    Therefore, the time complexity of searching without collision is \(O(1)\)
    and with collision is \(O(n)\).
\end{solution}
\section*{Problem 2}
\begin{solution}
    The algorithm is:
    \begin{enumerate}
        \item Let \(H\) be a hash table with \(k\) cells. The time complexity
        is \(O(k)\).
        \item For every element in \(L\): Insert the element to the \(k\)th
        cell where \(k\) is the key of the element. The time complexity is
        \(O(1)\). If the cell is not empty then chain the element. The time
        complexity is \(O(n)\).
        \item Make a new sorted \(L\) by appending the elements in every cell
        of \(H\) in order. We have \(n\) elements and \(k\) keys so there are
        at most \(\frac{n}{k}\) values in every cell of \(H\). The time
        complexity is \(O(k)\cdot O(\frac{n}{k})=O(k\cdot\frac{n}{k})=O(k)\).
    \end{enumerate}
    The time complexity of the algorithm is \(O(n+k)\).
\end{solution}
\section*{Problem 3}
\begin{solution}
    The algorithm using the sliding window approach is:
    \begin{enumerate}
        \item 
    \end{enumerate}
    The time complexity of the algorithm is \(O(n)\).
\end{solution}
\section*{Problem 4}
\begin{solution}
    The algorithm using the sliding window approach is:
    \begin{enumerate}
        \item 
    \end{enumerate}
    The time complexity of the algorithm is \(O(n)\).
\end{solution}
\end{document}