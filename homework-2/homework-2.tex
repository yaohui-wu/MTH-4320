\documentclass[12pt, letterpaper]{article}
\usepackage[margin=1in]{geometry}
\usepackage{amsmath}
\usepackage{amsthm}
\usepackage{amssymb}

\newenvironment{solution}{\begin{proof}[Solution]}{\end{proof}}
\renewcommand*{\qedsymbol}{\(\blacksquare\)}

\title{MTH 4320 Homework 2}
\author{Yaohui Wu}
\date{February 14, 2024}

\begin{document}
\maketitle
\section{Problem 1}
\begin{solution}
    \begin{align*}
        T(1) &= O(1) \\ T(n) &= 2\cdot T(n-1) \\ &= 2(2\cdot T(n-2)) \\ &= 2^n\cdot T(n-(n-1)) \\
        &= 2^n\cdot T(1)) \\ &= O(2^n)
    \end{align*}
    The running time is
    \begin{align*}
        T(1) &= O(1) \\ T(n) &= O(2^n)
    \end{align*}
\end{solution}
\section{Problem 2}
\begin{solution}
    We have
    \begin{align*}
        T(1) &= O(1) \\
        T(n) &= n+4\cdot T\left(\frac{n}{2}\right) \\ &= n+4\left[\frac{n}{2}+4\cdot T\left(\frac{n}{2^2}\right)\right] \\
        &= n+2n+4^2\cdot T\left(\frac{n}{2^2}\right) \\ &= n+2n+\cdots+2^{k-1}n+4^k\cdot T\left(\frac{n}{2^k}\right)
    \end{align*}
    \(T(n)\) converges when \(T\left(\frac{n}{2^k}\right)=T(1)\) so \(k=\log n\) then we have
    \begin{align*}
        T(n) &= n+2n+2^2n+\cdots+2^{\log(n)-1}n+4^{\log n}T(1) \\ &= n+2n+2^2n+\cdots+\frac{n^2}{2}+n^2\cdot T(1) \\ &= O(n^2)
    \end{align*}
    The running time is
    \begin{align*}
        T(1) &= O(1) \\ T(n) &= O(n^2)
    \end{align*}
\end{solution}
\section{Problem 3}
\begin{solution}
    The algorithm is
    \begin{enumerate}
        \item Sort \(A\) in ascending order using merge sort.
        \item Find the set of all sums of pairs of elements from \(A\).
        \item Iterate over the pairs and use binary search to find if there is another pair that sums to \(2000\).
    \end{enumerate}
    The running time is \(O(n\log n)+O(n^2)+O(n^2\log n)=O(n^2\log n)\).
\end{solution}
\section{Problem 4}
\begin{solution}
    The algorithm is
    \begin{enumerate}
        \item Divide \(L\) into two intervals \(L\) and \(R\) of the same size. 
        \item Find the largest number in \(L\) and \(R\) using recursion.
        \item Return the largest number in each interval then compare them to find the largest number in \(L\).
    \end{enumerate}
    The running time is \(O(n)\).
\end{solution}
\section{Problem 5}
\begin{solution}
    The algorithm is
    \begin{enumerate}
        \item Divide \(A\) into  two intervals \(L\) and \(R\) of equal size.
        \item Sort the elements in each interval and iterate over the elements of \(L\) and \(R\).
        \item If the current element in \(L\) is greater than an element in \(R\)
        then we add the number of the remaining elements in \(L\) to the count of the flipped pairs.
        \item If the current element in \(L\) is not greater than an element in \(R\) then we skip it
        and we make sure we only count every flipped pair exactly once.
        \item Merge \(L\) and \(R\) then return the sum of the count of the flipped pairs using recursion.
    \end{enumerate}
    The running time is \(O(n\log n)\).
\end{solution}
\end{document}