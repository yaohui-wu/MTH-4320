\documentclass[12pt]{article}
\usepackage[letterpaper, margin=1in]{geometry}
\usepackage{mlmodern}
\usepackage{amsmath}
\usepackage{amsthm}
\usepackage{amssymb}

\newenvironment*{solution}{\begin{proof}[Solution]}{\end{proof}}
\renewcommand*{\qedsymbol}{\(\blacksquare\)}

\title{MTH 4320 Homework 9}
\author{Yaohui Wu}
\date{April 19, 2024}

\begin{document}
\maketitle
\section*{Problem 1}
\begin{solution}
    Let \(P[i]\) be the length of the longest primish subsequence \(S[i]\)
    that ends with element \(L[i]\) for \(i=1,2,\dots,n\). We have
    \begin{align*}
        P[1] &= 1 \\
        P[2] &= \begin{cases}
            1+P[1], &\text{if \(S[1]+L[2]\) is prime} \\
            1, &\text{otherwise}
        \end{cases} \\
        P[3] &= \begin{cases}
            1+\max_{1\leq j<3}P[j], &\text{if \(S[j]+L[3]\) is prime} \\
            1, &\text{otherwise}
        \end{cases}
    \end{align*}
    In general, we can deduce that
    \begin{align*}
        P[i] &= \begin{cases}
            1+\max_{1\leq j<i}P[j], &\text{if \(S[j]+L[i]\) is prime} \\
            1, &\text{otherwise}
        \end{cases}
    \end{align*}
    The algorithm returns the length of the longest primish subsequence in
    \(L\) by computing \[P=\max_{1\leq i\leq n}P[i]\] Assuming we can check if
    a number is prime in \(O(1)\) time, computing \(P[i]\) takes \(O(i)\) time
    so computing \(P\) takes \(O(1)+O(2)+\dots+O(n)=O(n^2)\) time. The time
    complexity of the algorithm is \(O(n^2)\).
\end{solution}
\section*{Problem 2}
\begin{solution}
    Let \(P(i,j)\) be the length of the longest palindrome in substring
    \(S[i,j]\) that starts with the \(i\)th letter and ends with \(j\)th
    letter. We have
    \begin{align*}
        P(i,j) &= \begin{cases}
            1, &\text{if \(i=j\)} \\
            2+P(i+1,j-1), &\text{if \(S[i]=S[j]\)} \\
            \max\{P(i,j-1),P(i+1,j)\}, &\text{otherwise}
        \end{cases}
    \end{align*}
    For \(i=1,2,\dots,n\) and \(j=1,2,\dots,n\), we compute each \(P(i,j)\) in
    \(O(1)\) time so the time complexity of the algorithm is \(O(n^2)\).
\end{solution}
\section*{Problem 3}
\begin{solution}
    The time complexity of the algorithm is \(O(n^2)\).
\end{solution}
\end{document}